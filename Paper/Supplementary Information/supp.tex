\documentclass[fleqn,12pt]{supp}
%%\usepackage{setspace}
%%\doublespacing
\usepackage{soul, float, hyperref}
\hypersetup{hidelinks}
\usepackage[utf8]{inputenc}

\newcommand{\beginsupplement}{%
        \setcounter{table}{0}
        \renewcommand{\thetable}{S\arabic{table}}%
        \setcounter{figure}{0}
        \renewcommand{\thefigure}{S\arabic{figure}}%
     }

\title{Supplementary Information}
\author{}

\begin{document}

\flushbottom
\maketitle

\vspace{-35pt}
\section*{A Structural and Functional Bioinformatics Study of QTY-designed Retinylidene Proteins}

\subsection*{Siqi Pan}

 

\begin{figure}[H]
    \caption{\textbf{The enlarged protein sequence alignments of 12 retinylidene proteins and their QTY analogs from Figure 1. } The symbols $|$ and $*$ indicate that amino acids are identical or different, respectively. Amino acids L, I/V, and F in TM (transmembrane) alpha helices (shown in blue above the sequences) are replaced with Q, T, and Y respectively. The variation in the TM domain ranges from 35.53\% to 50.24\% while the overall variation rate ranges from 15.48\% to 30.04\%. 
    The alignments are 
    \textbf{a)} OPN1MW \textit{vs} OPN1MW$^{\textrm{QTY}}$, 
    \textbf{b)} OPN1LW \textit{vs} OPN1LW$^{\textrm{QTY}}$, 
    \textbf{c)} OPN1SW \textit{vs} OPN1SW$^{\textrm{QTY}}$, 
    \textbf{d)} OPN2 \textit{vs} OPN2$^{\textrm{QTY}}$, 
    \textbf{e)} OPN3 \textit{vs} OPN3$^{\textrm{QTY}}$, 
    \textbf{f)} OPN4 \textit{vs} OPN4$^{\textrm{QTY}}$, 
    \textbf{g)} OPN5 \textit{vs} OPN5$^{\textrm{QTY}}$, 
    \textbf{h)} RGR \textit{vs} RGR$^{\textrm{QTY}}$, 
    \textbf{i)} RRH \textit{vs} RRH$^{\textrm{QTY}}$, 
    \textbf{j)} BACR \textit{vs} BACR$^{\textrm{QTY}}$, 
    \textbf{k)} BACH \textit{vs} BACH$^{\textrm{QTY}}$, 
    \textbf{l)} ChR2 \textit{vs} ChR2$^{\textrm{QTY}}$. }
    \textbf{a)} OPN1MW \textit{vs} OPN1MW$^{\textrm{QTY}}$ \\
    \includegraphics[width=\linewidth]{FigureS1a.jpg}
\end{figure}

\newpage
\begin{figure}[H]
    \textbf{b)} OPN1LW \textit{vs} OPN1LW$^{\textrm{QTY}}$ \\
    \includegraphics[width=\linewidth]{FigureS1b.jpg}
\end{figure}

\newpage
\begin{figure}[H]
    \textbf{c)} OPN1SW \textit{vs} OPN1SW$^{\textrm{QTY}}$ \\
    \includegraphics[width=\linewidth]{FigureS1c.jpg}
\end{figure}

\newpage
\begin{figure}[H]
    \textbf{d)} OPN2 \textit{vs} OPN2$^{\textrm{QTY}}$ \\
    \includegraphics[width=\linewidth]{FigureS1d.jpg}
\end{figure}

\newpage
\begin{figure}[H]
    \textbf{e)} OPN3 \textit{vs} OPN3$^{\textrm{QTY}}$ \\
    \includegraphics[width=\linewidth]{FigureS1e.jpg}
\end{figure}

\newpage
\begin{figure}[H]
    \textbf{f)} OPN4 \textit{vs} OPN4$^{\textrm{QTY}}$ \\
    \includegraphics[width=\linewidth]{FigureS1f.jpg}
\end{figure}

\newpage
\begin{figure}[H]
    \textbf{g)} OPN5 \textit{vs} OPN5$^{\textrm{QTY}}$ \\
    \includegraphics[width=\linewidth]{FigureS1g.jpg}
\end{figure}

\newpage
\begin{figure}[H]
    \textbf{h)} RGR \textit{vs} RGR$^{\textrm{QTY}}$ \\
    \includegraphics[width=\linewidth]{FigureS1h.jpg}
\end{figure}

\newpage
\begin{figure}[H]
    \textbf{i)} RRH \textit{vs} RRH$^{\textrm{QTY}}$ \\
    \includegraphics[width=\linewidth]{FigureS1i.jpg}
\end{figure}

\newpage
\begin{figure}[H]
    \textbf{j)} BACR \textit{vs} BACR$^{\textrm{QTY}}$ \\
    \includegraphics[width=\linewidth]{FigureS1j.jpg}
\end{figure}

\newpage
\begin{figure}[H]
    \textbf{k)} BACH \textit{vs} BACH$^{\textrm{QTY}}$ \\
    \includegraphics[width=\linewidth]{FigureS1k.jpg}
\end{figure}

\newpage
\begin{figure}[H]
    \textbf{l)} ChR2 \textit{vs} ChR2$^{\textrm{QTY}}$ \\
    \includegraphics[width=\linewidth]{FigureS1l.jpg}
\end{figure}



\begin{figure}[H]
    \caption{\textbf{AlphaFold3 prediction accuracy: plDDT, PAE, ipTM, and pTM scores.} \\
    The scores are displayed for the following proteins: 
    \textbf{a)} OPN1MW$^{\textrm{QTY}}$,
    \textbf{b)} OPN1LW$^{\textrm{QTY}}$,
    \textbf{c)} OPN1SW$^{\textrm{QTY}}$,
    \textbf{d)} OPN2$^{\textrm{QTY}}$,
    \textbf{e)} OPN3$^{\textrm{QTY}}$,
    \textbf{f)} OPN4$^{\textrm{QTY}}$,
    \textbf{g)} OPN5$^{\textrm{QTY}}$,
    \textbf{h)} RGR$^{\textrm{QTY}}$,
    \textbf{i)} RRH$^{\textrm{QTY}}$,
    \textbf{j)} BACR$^{\textrm{QTY}}$ monomer,
    \textbf{k)} BACH$^{\textrm{QTY}}$ monomer,
    \textbf{l)} ChR2$^{\textrm{QTY}}$ monomer,
    \textbf{m)} BACR$^{\textrm{QTY}}$ trimer,
    \textbf{n)} BACH$^{\textrm{QTY}}$ trimer,
    \textbf{o)} ChR2$^{\textrm{QTY}}$ dimer.}
    \textbf{a)} OPN1MW$^{\textrm{QTY}}$ \\
    \includegraphics[width=\linewidth]{FigureS2a.jpg}
\end{figure}

\newpage
\begin{figure}[H]
    \textbf{b)} OPN1LW$^{\textrm{QTY}}$ \\
    \includegraphics[width=\linewidth]{FigureS2b.jpg}
\end{figure}

\newpage
\begin{figure}[H]
    \textbf{c)} OPN1SW$^{\textrm{QTY}}$ \\
    \includegraphics[width=\linewidth]{FigureS2c.jpg}
\end{figure}

\newpage
\begin{figure}[H]
    \textbf{d)} OPN2$^{\textrm{QTY}}$ \\
    \includegraphics[width=\linewidth]{FigureS2d.jpg}
\end{figure}

\newpage
\begin{figure}[H]
    \textbf{e)} OPN3$^{\textrm{QTY}}$ \\
    \includegraphics[width=\linewidth]{FigureS2e.jpg}
\end{figure}

\newpage
\begin{figure}[H]
    \textbf{f)} OPN4$^{\textrm{QTY}}$ \\
    \includegraphics[width=\linewidth]{FigureS2f.jpg}
\end{figure}

\newpage
\begin{figure}[H]
    \textbf{g)} OPN5$^{\textrm{QTY}}$ \\
    \includegraphics[width=\linewidth]{FigureS2g.jpg}
\end{figure}

\newpage
\begin{figure}[H]
    \textbf{h)} RGR$^{\textrm{QTY}}$ \\
    \includegraphics[width=\linewidth]{FigureS2h.jpg}
\end{figure}

\newpage
\begin{figure}[H]
    \textbf{i)} RRH$^{\textrm{QTY}}$ \\
    \includegraphics[width=\linewidth]{FigureS2i.jpg}
\end{figure}

\newpage
\begin{figure}[H]
    \textbf{j)} BACR$^{\textrm{QTY}}$ monomer \\
    \includegraphics[width=\linewidth]{FigureS2j.jpg}
\end{figure}

\newpage
\begin{figure}[H]
    \textbf{k)} BACH$^{\textrm{QTY}}$ monomer \\
    \includegraphics[width=\linewidth]{FigureS2k.jpg}
\end{figure}

\newpage
\begin{figure}[H]
    \textbf{l)} ChR2$^{\textrm{QTY}}$ monomer \\
    \includegraphics[width=\linewidth]{FigureS2l.jpg}
\end{figure}

\newpage
\begin{figure}[H]
    \textbf{m)} BACR$^{\textrm{QTY}}$ trimer \\
    \includegraphics[width=\linewidth]{FigureS2m.jpg}
\end{figure}

\newpage
\begin{figure}[H]
    \textbf{n)} BACH$^{\textrm{QTY}}$ trimer \\
    \includegraphics[width=\linewidth]{FigureS2n.jpg}
\end{figure}

\newpage
\begin{figure}[H]
    \textbf{o)} ChR2$^{\textrm{QTY}}$ dimer \\
    \includegraphics[width=\linewidth]{FigureS2o.jpg}
\end{figure}

\begin{figure}[H]
    \caption{\textbf{The radius of gyration of native and QTY-designed OPN2. } By convention, the isomerization is set at time 0ns, which is indicated by a brown, vertical dashed line. The radius of gyration and its 1-ns running average are shown, respectively, as black and red lines. }
    \textbf{a)} Native OPN2 \\ \\
    \includegraphics[width=\linewidth]{FigureS3a.jpg}
    \textbf{b)} QTY-designed OPN2 \\ \\
    \includegraphics[width=\linewidth]{FigureS3b.jpg}
\end{figure}

\begin{figure}[H]
    \caption{\textbf{The root mean square (RMS) fluctuation of native and QTY-designed OPN2. } Blue bars represent proteins with 11-cis-retinal and yellow bars represent proteins with all-trans-retinal. The two states are plotted on the same set of axis, with the shorter bars placed on top of the taller bars. }
    \textbf{a)} Native OPN2 \\ \\
    \includegraphics[width=\linewidth]{FigureS4a.jpg}
    \textbf{b)} QTY-designed OPN2 \\ \\
    \includegraphics[width=\linewidth]{FigureS4b.jpg}
\end{figure}

\begin{figure}[H]
    \caption{\textbf{The root mean square distance (RMSD) of each residue in the retinal-binding pocket of native and QTY-designed OPN2. } By convention, the isomerization is set at time 0ns, which is indicated by a brown, vertical dashed line. Only the 1-ns running average of RMSD are shown for clarity. }
    \textbf{a)} Native OPN2 \\ \\
    \includegraphics[width=\linewidth]{FigureS5a.jpg}
    \textbf{b)} QTY-designed OPN2 \\ \\
    \includegraphics[width=\linewidth]{FigureS5b.jpg}
    
    \vspace{0.5cm}
    \includegraphics[width=\linewidth]{FigureS5legend.jpg}
\end{figure}

\end{document}